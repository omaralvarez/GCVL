%
% Resumen del proyecto de fin de carrera
%

\section*{Abstract:}

The objective of this project is designing and implementing a multi-platform HPC Stereo Matching solution. Any technique that is capable of collecting 3D information and creating the illusion of depth in an image, is considered stereoscopic vision or tridimensional (3D) vision. Our vision is stereoscopic by nature, being able to perceive the sensation of depth, distance, etc. This is achieved thanks to the horizontal separation between our eyes, leading to the processing of the differences between the perceived images of the visual system by our brain.

Artificial systems for stereoscopic vision for the obtaining of 3D information in multiple applications, have been employed for several decades. The main problem that these systems tackle; is the determination of the correspondence between the pixels that come from the same point in the image pairs of the tridimensional scene, or Stereo Matching.

After a preliminary phase for the study of the current algorithms for Stereo Matching, a base sequential implementation using \CC will be developed. This multi-paradigm language extends the amply known programming language C with mechanisms that allow the manipulation of objects or the ability for using generic programming. This compromise between efficiency and versatility makes this language vastly used in HPC environments.

For the parallel versions of the code in CPUs we will utilize OpenMP, a multi-process programming API for shared memory platforms. Moreover, for the GPU implementation, GPGPU technologies will be used. CUDA and OpenCL are APIs that allow the creation of applications with data-level and task-level parallelism that can be executed in graphical processors.  





