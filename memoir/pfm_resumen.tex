%
% Resumen del proyecto de fin de carrera
%

\section*{Abstract:}

The objective of this project is designing and implementing a multi-platform point-based rendering visualizer that allows the management and interactive manipulation of big 3D point clouds. These clouds can be obtained using a LIDAR scanner or cheaper alternatives like KINECT. 

These systems can nowadays generate data-sets with huge amounts of point data, position, color, normals, reflectivity, etc. As a consequence, these point clouds will not fit on system RAM and will have to reside in an HDD. Due to this fact, real-time rendering of massive point clouds is a complex task.

The resulting software tool developed in this project, will offer the end user the functionality necessary for working with these types of datasets. Including a complete 3D visualizer, with advanced OpenGL point-rendering capabilities, multiple point cloud support, multiple cameras, multi-resolution capabilities, point cloud transformation and preprocessing, distance measurements, object segmentation, CAD exportation of segmented objects, etc. 

To implement all of these features, we will utilize PCM. PCM is a project in development from the University of A Coru�a, that provides a series of low level tools for working with point clouds on commodity hardware. This project will not only extend PCM with the aforementioned functionality and a high level software tool for the end user, but will improve the existing code and some performance aspects of PCM.

Furthermore, the provided tool will take advantage of all the parallel computing capabilities of the target platforms. Using multiple cores on CPUs and GPGPU on the graphics cards. The multi-resolution framework will also help when trying to achieve maximum performance when trying to apply any computation.

In conclusion, the resulting software package should be easily integrated and improve substantially the existing engineering workflows, thanks to the fact that the tool will be multi-platform and open source. Since it was also developed with the advice of some surveyors that are working with laser scanners everyday, there are high hopes of truly having created a versatile and useful tool.




