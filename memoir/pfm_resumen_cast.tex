%
% Resumen del proyecto de fin de carrera
%

\section*{Resumen:}

El objetivo de este proyecto es dise�ar e implementar un motor de render basado en puntos que utilice ray tracing como m�todo de iluminaci�n global. 

El render basado en puntos tiene m�ltiples ventajas con respecto al cl�sico que utiliza mallas de pol�gonos, sobre todo en el caso en el que los pol�gonos puedan llegar a ser m�s peque�os que un pixel. 

En este proyecto se abordar� la representaci�n de modelos de puntos, tanto sin t�cnicas de ray tracing (rasterizado normal), as� como mediante ray tracing que ser� el verdadero objetivo del proyecto.

Tambi�n es importante se�alar que debido a la naturaleza de las nubes de puntos que tienden a ser muy grandes, para la aceleraci�n del proceso de ray tracing se utilizar� tambi�n un k-d tree. Esto es una estructura de datos de particionado espacial para la organizaci�n de las nubes de puntos. Esta estructura permite reducir sustancialmente el tiempo de render, sin ella el proceso de ray tracing ser�a demasiado costoso y el render llevar�a demasiado tiempo.

El render por puntos tambi�n tiene algunas desventajas, como por ejemplo la naturaleza adimensional de los mismos. Esto quiere decir que el punto no tiene superficie, volumen o normal. Esto da lugar a problemas a la hora de hacer el render. En este proyecto se exploran diversas soluciones para estos problemas.

Para facilitar al usuario final la utilizaci�n del proyecto, as� como tambi�n hacer m�s f�cil el testing; se programar� tambi�n un plugin compatible con Blender que permitir� el renderizado de escenas creadas en blender mediante el proyecto y adem�s facilitar� el uso del motor gracias a la integraci�n del mismo con la interfaz de Blender.  


