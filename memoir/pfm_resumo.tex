%
% Resumen del proyecto de fin de carrera
%

\section*{Resumo:}

O obxectivo deste proxecto � o dese�o e implementaci�n dunha
ferramenta multiplataforma para Stereo Matching. Consideramos como visi�n estereosc�pica ou visi�n tridimensional (3D) a calquera t�cnica capaz de recoller informaci�n visual tridimensional e crear ilusi�n de profundidade nunha imaxe. De maneira natural a nosa visi�n � estereosc�pica, puidendo percibir sensaci�n de profundidade, afastamento, etc. Isto se consegue grazas � separaci�n horizontal dos nosos ollos, procesando o noso cerebro as diferenzas entre as imaxes percibidas polo noso sistema visual.

V��ense empregando sistemas artificiais de visi�n estereosc�pica para a obtenci�n de informaci�n 3D en diferentes aplicaci�ns dende hai varias d�cadas. O problema central que abordan estes sistemas � o da determinaci�n da correspondencia entre os p�xels que prove�en do mesmo punto dos pares de imaxes da escena tridimensional, ou correspondencia estereosc�pica.

Tras unha primeira fase de estudo dos distintos algoritmos existentes, levarase a cabo unha implementaci�n secuencial base empregando a linguaxe de programaci�n \CC. Esta linguaxe multiparadigma estende a ampliamente co�ecida liguaxe de programaci�n C con mecanismos que permiten tanto a manipulaci�n de obxectos como a capacidade para a programaci�n xen�rica. Este compromiso entre eficiencia e versatilidade fai que esta linguaxe sexa ampliamente utilizada en entornos HPC.

Para as versi�ns paralelas do c�digo en CPU utilizarase OpenMP, unha API de programaci�n multi-proceso en plataformas de memoria compartida. En canto � implementaci�n GPU, levarase a cabo cunhas tecnolox�as GPGPU, CUDA e OpenCL, unhas APIs que permiten crear aplicaci�ns con paralelismo a nivel de datos e tarefas que poden executarse en procesadores gr�ficos.


