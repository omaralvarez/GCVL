%
% T�TULO DEL CAP�TULO
%
\chapter{GCVL: GPU Computer Vision Library
	\label{chapter_3}
}

Intro

\section{Development Methodology}

Agile software development \cite{agiledev} is a combination of development methods that use iterative and incremental development, where requirements and solutions mature through collaboration between self-organizing, cross-functional teams. The motto of this method is ``embrace change''; that is why it encourages adaptive planning, evolutionary development and delivery, a time-boxed iterative approach, and promotes quick and flexible response to change.

\subsection{Agile manifesto}

In February of 2001, several developers met at Snowbird, Utah resort, to debate different lightweight development methods. They published the \emph{Manifesto for Agile Software Development} \cite{agilemani} to define the approach that is now called agile software development. 

The conclusions that we can reach from the manifesto's items are described below:

\begin{itemize}
\item \textbf{Individuals and Interactions}: In agile development self-organization and motivation are really important. Other values promoted by the manifesto are co-location\footnote{The act of placing multiple individuals within a single location.} and pair programming\footnote{Two programmers work together at one workstation.}.
\item \textbf{Working software}: Working software will be utilized for more purposes than presenting documents to the client.
\item \textbf{Customer collaboration}: The software requirements cannot be fully realized from the beginning of the software development cycle, so being in touch with the customer is really important.
\item \textbf{Responding to change}: Agile development is keen on fast responses to change and continuous development.  
\end{itemize}

More principles are mentioned in the manifesto, some of them are:

\begin{itemize}
\item Customer satisfaction by rapid delivery of useful software.
\item Welcome changes even late in the development.
\item Working software is the principal measure of progress.
\item Maintaining a constant pace.
\item Cooperation between business people and developers. 
\item Build projects around motivated individuals.
\item Attention to technical excellence.
\item Simplicity.
\end{itemize}

Agile methods break down tasks into small increments with minimal planning and normally long-term planning is not directly involved. \emph{Iterations} are short timeframes that typically last from one to four weeks. A team works in each iteration through a full software development cycle; including planning, requirements analysis, design, coding, etc. This minimizes risk and facilitates adaptation to change. An iteration may not add enough new functionalities to warrant a market release, but the objective is to have an available release at the end of each iteration. 

Team composition does not depend on corporate hierarchies or corporate roles of team members. They normally have the responsability of completing tasks that deliver the required functionalities that an iteration requires. How to meet an iteration's objectives is decided individually.

The ``weight'' of the method depends on the type of project, the planning and order of tasks in a generalist project should not be the same as in a research project.  

Agile methods encourage face-to-face communication instead of written documents if possible. Most teams work in an open office (the \emph{bullpen}), which makes this type of communication easier. 

Each agile team contains a customer representative, that ensures that customer needs and company goals are aligned. 

Most agile methods encourage a routine that includes daily face-to-face communication among team members. In a brief session team members tell each other what they achieved the previous day, what they are going to do today and the problems that have appeared. 

As agile development emphasizes on working software as the primary measure of progress and has a clear preference in face-to-face communication this results in less written documentation than other methods. This does not mean that documentation should be disregarded, but that less emphasis is made on documentation because is not needed as much.

\section{Technology}

Tecnologia usada?

\section{Design}

Diagrama clases general en intro

\subsection{General Tools}

Config y demas

\subsection{CPU Module}

cpu

\subsection{OpenCL Module}

opencl

\subsection{CUDA Module}

cuda


